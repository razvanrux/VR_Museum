\chapter{Testarea aplicației și rezultate experimentale}
\label{cap:cap4}

(aproximativ 7 pagini)

\begin{itemize}
    \item Punerea în funcțiune/lansarea aplicației,elemente de configurare sau instalare;
    \item Testarea sistemului (hardware/software);
    \item Aspecte legate de încărcarea procesorului, memoriei, limitări în ce privește transmisia datelor/comunicarea;
    \item Se prezintă datele de test/metrici/benchmarks
    \item Aspecte legate de fiabilitate/securitate/scalabilitate;
    \item Rezultate experimentale (grafice, tabele, etc.);
    \item Utilizarea sistemului. 
\end{itemize}

\section{Nivel 1}
\label{cap:cap4:nivel1}

\textcolor{gray}{\lipsum}  (Figura \ref{fig:clone_trooper})

\begin{figure}[H]
    \centering
    \includegraphics[width=0.3\textwidth]{continut/capitol4/figuri/clone_trooper.png}
    \caption{Clone Trooper\protect\footnotemark}
    \label{fig:clone_trooper}
\end{figure}
\footnotetext{imagine preluată de pe un site web care nu „merită” trecut la bibliografie \url{https://www.pngitem.com/}}

Tabelul \ref{tabel:tab_simplu} prezintă un exemplu foarte simplu de tabel în \LaTeX. Un alt exemplu este prezentat în Tabelul \ref{tabel:2} din \ref{anexa3:func_xyz}.

\begin{table}[ht]
    \centering
    \caption{Exemplu de tabel simplu}
    \label{tabel:tab_simplu}
    \begin{tabular}{||c l c r||} 
        \hline
        \textbf{Col1} & \textbf{Col2} & \textbf{Col3} & \textbf{Col4} \\
        \hline\hline
        1 & 6 & 87837 & 787 \\ 
        2 & 7 & 78 & 5415 \\
        3 & 545 & 778 & 7507 \\
        4 & 545 & 18744 & 7560 \\
        5 & 88 & 788 & 6344 \\
        \hline
    \end{tabular}
\end{table}

\subsection{Nivel 2}
\label{cap:cap4:nivel1:nivel2}

\subsubsection{Nivel 3}
\label{cap:cap4:nivel1:nivel2:nivel3}

\textcolor{gray}{\lipsum}

\textcolor{gray}{\lipsum}

\textcolor{gray}{\lipsum}
