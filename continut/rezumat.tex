\newpage
\begin{center}
    \LARGE
    \textbf{\thesistitle}
    
    \vspace{0.5cm}
    
    \authornamefl
    
    \vspace{1cm}
    
    \textbf{Rezumat}
    
    \vspace{1cm}
\end{center}

\textbf{Această secțiune va fi dezvoltată odată cu finalizarea proiectului de licență, cât și a documentației}

Proiectul de diplomă/lucrarea de disertație va conține un scurt rezumat (abstract – termenul în engleză) de maximum o pagină, care expune ideile principale ale tezei, contribuția autorului la realizarea temei propuse precum și concluziile obținute. Formulați ideile principale ale lucrării în propoziții sau fraze cât mai simple, fără amănunte nesemnificative. Încercați să atingeți următoarele idei:
\begin{itemize}
    \item contextul și motivația alegerii temei;
    \item obiectivele principale;
    \item metodologia utilizată;
    \item soluția propusă;
    \item rezultatele esențiale și principalele concluzii.
\end{itemize}

Rezumatul ar trebui să fie între 150-250 de cuvinte și să fie urmat de cuvinte-cheie (\textit{keywords}, 5-7 termeni relevanți).

\textit{Recomandare neoficială: Rezumatul se compune, de obicei, după ce ați terminat redactarea lucrării de licență/disertație, după finalizarea capitolului Concluzii}