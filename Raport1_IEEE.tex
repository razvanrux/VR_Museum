\documentclass[conference, 10pt]{IEEEtran}
\IEEEoverridecommandlockouts

\usepackage[romanian]{babel}
\selectlanguage{romanian}

\usepackage{cite}
\usepackage{amsmath,amssymb,amsfonts}
\usepackage{algorithmic}
\usepackage{graphicx}
\usepackage{textcomp}
\usepackage{xcolor}

\usepackage[unicode, hidelinks, colorlinks, linkcolor=black, urlcolor=blue, citecolor=black]{hyperref}

\usepackage{flushend}

\def\BibTeX{{\rm B\kern-.05em{\sc i\kern-.025em b}\kern-.08em
    T\kern-.1667em\lower.7ex\hbox{E}\kern-.125emX}}

%% date caracteristice universitatii/facultatii.
%% in acest fisier nu trebuie aduse modificari
%% comune: coperta externa si interna
\newcommand{\university}    {Universitatea Tehnică „Gheorghe Asachi” din Iași}
\newcommand{\faculty}       {Facultatea de Automatică și Calculatoare}
\newcommand{\location}      {Iași}
%% datele specifice: nume, prenume student; titlu lucrare; nume, prenume, grad didactic coordonator
%% PENTRU MODIFICARE SE EDITEAZA FISIERUL macros/student.tex
%% date student si lucrare
\newcommand{\thesistitle}   {Explorarea Muzeelor prin Realitate Virtuală}    %<---------
\newcommand{\authorlast}    {Ruxandari}         %<---------
\newcommand{\authorfirst}   {Răzvan}
\newcommand{\authornamefl}  {\authorfirst \space \authorlast} % first name last
\newcommand{\authornamelf}  {\authorlast \space \authorfirst} % last name first

%% titlul academic si numele coordonatorului stiintific
\newcommand{\coordinator}   {Ș.l.dr.ing. Otilia ZVORIȘTEANU}
%% pentru numele complet si grad didactic/titlu academic, consultati https://ac.tuiasi.ro/despre/personal/cadre-didactice-ale-departamentului-de-calculatoare/

\newcommand{\studyfieldlbl} {Domeniul: }
\newcommand{\studyfield}    {Calculatoare și Tehnologia Informației}
\newcommand{\studyproglbl}  {Programul de studii: }
\newcommand{\studyprog}     {Tehnologia Informației}
\newcommand{\promotion}     {2025}

%% absolventii de master vor inlocui "diploma" cu "disertatie"
\newcommand{\thesistype}    {Proiect de diplomă}
%\newcommand{\thesistype}    {Lucrare de disertație}

%% suplimentar fata de lista de pachete incluse, pot fi adaugate si altele, in functie de specificul tezei de licenta

\begin{document}

% poate fi eventual interschimbat: Raport nr. 1 \\ \thesistitle
\title{\thesistitle \\
    {\Large \textbf{Raport nr. 1}}
}

\author{
    \IEEEauthorblockN{\authornamefl}
    \IEEEauthorblockA{
        \faculty \\
        \university \\ 
        \studyfieldlbl \studyfield \\ 
        \studyproglbl \studyprog \\
        \url{email@student.tuiasi.ro}
    }
    \and
    \IEEEauthorblockN{Coordonator: \coordinator}
    \IEEEauthorblockA{
        \faculty \\
        \university \\ 
        \url{email@academic.tuiasi.ro}
    }
}

\maketitle

\begin{abstract}
    Rezumatul poate să includă o descriere sumară a domeniului în care se încadrează teza. Recomandarea ar fi să încercați să acoperiți capitolul \textit{Introducere} și, respectiv, \textit{Capitolul 1} al tezei finale.
\end{abstract}

Primul raport trebuie să aibă maxim 6 pagini și să conțină:

\section{Introducere}
\begin{itemize}
    \item Contextul general.
    \item Motivarea alegerii temei și relevanța în domeniu.
\end{itemize}

\section{Stadiul Actual}
\begin{itemize}
    \item Analiza situației curente din domeniu:
    \begin{itemize}
        \item Rezumat al literaturii de specialitate relevante.
        \item Soluții existente (sisteme, metode, tehnologii) și limitările acestora.
    \end{itemize}
    \item Identificarea golurilor de cercetare: Ce lipsește în soluțiile actuale sau ce ar putea fi îmbunătățit.
    \item Context local versus internațional (dacă este aplicabil).
\end{itemize}

\section{Obiective (Se preiau din raportul îndrumătorului)}
\begin{itemize}
    \item Scopul general al proiectului.
    \item Obiective specifice: Ce anume își propune proiectul să adreseze în contextul descris la \textit{Stadiul Actual}.
\end{itemize}

\section{Metodologie}
\begin{itemize}
    \item Resurse utilizate: Hardware și software, medii de dezvoltare.
    \item Metode și algoritmi: Descriere succintă și justificare pentru alegerea lor.
\end{itemize}

\section{Activități Desfășurate până în prezent}
\begin{itemize}
    \item Activitățile realizate până la momentul redactării raportului:
    \begin{itemize}
        \item Documentare, analize și experimentări.
        \item Progrese sau prototipuri (dacă există).
    \end{itemize}
    \item Probleme întâmpinate și cum au fost abordate.
\end{itemize}

\section{Rezultate corelate cu obiectivele semestrului I}
\begin{itemize}
    \item Corelarea rezultatelor obținute/preconizate cu obiectivele stabilite de îndrumător.
    \item Ce se preconizează să fie obținut în etapa următoare:
    \begin{itemize}
        \item Contribuția proiectului în domeniu.
        \item Rezultate tangibile (ex. prototipuri, analize).
    \end{itemize}
\end{itemize}

\section{Concluzii Preliminare}
\begin{itemize}
    \item Sinteza progresului actual.
    \item Pași următori pentru îndeplinirea obiectivelor proiectului.
\end{itemize}

\section{Bibliografie}
\begin{itemize}
    \item Lista de referințe utilizate pentru stadiul actual și alte secțiuni. Referințe vor fi introduse în \texttt{bibliografie/bibliografie.bib} și referite prin comanda \texttt{\\cite\{label\}} \cite{misc:web:rfc7231}. Referințele care includ în câmpul \texttt{note} comanda \texttt{\\doi\{\}} vor fi trecute în dublu exemplar:
    \begin{enumerate}
        \item o intrare simplă, fără \texttt{note} -- va fi utilizată în cadrul raportului și în cadrul articolului pentru SCSS;
        \item o intrare cu \texttt{note} (modelul exemplificat în template) -- va fi utilizată în cadrul lucrării finale de licență.
    \end{enumerate}
\end{itemize}

\section{Specificații finale}

Acest raport trebuie să aibă maxim 6 pagini și să fie structurat folosind stilul IEEE, care include:
\begin{itemize}
    \item Două coloane pe pagină.
    \item Formatare standard (font Times New Roman, dimensiune 10pt, spațiere standard IEEE).
\end{itemize}
Formatul IEEE este folosit la Sesiunea de Comunicări Științifice Studențești și la conferința ICSTCC.

Template-uri în Word și LaTeX se găsesc la adresa:  
\url{https://www.ieee.org/conferences/publishing/templates.html}

Formularul de încărcare a documentului va fi transmis la începutul lunii ianuarie.  
\textbf{Termenul de încărcare} este \textbf{\textit{\textcolor{red}{\textit{27 ianuarie 2025, ora 23:59}}}}.

% *************** referinte bibliografice ***************
\nopagebreak
\bibliographystyle{IEEEtran}
\bibliography{bibliografie/bibliografie}

\end{document}